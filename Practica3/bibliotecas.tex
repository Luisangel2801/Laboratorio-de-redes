\usepackage[spanish]{babel}
\usepackage[utf8]{inputenc}
\usepackage[hidelinks]{hyperref}
\usepackage[letterpaper, margin=1in, nohead]{geometry}
\usepackage{longtable}
\usepackage{booktabs}
\usepackage{graphicx}
\usepackage{pgfgantt}
\usepackage{soul}
\usepackage[bottom]{footmisc}
\usepackage{titlesec}
\titleformat*{\section}{\large\bfseries}
\titleformat*{\subsection}{\normalsize\bfseries}
\usepackage{palatino}
\usepackage{caption}
\usepackage{imakeidx}
\makeindex[columns=3, title=Alphabetical Index, intoc]
\usepackage{float}
% --- Para el encabezado y pie de página
%\usepackage{fancyhdr}
%\pagestyle{fancy}
%\lhead[]{}
%\chead[]{}
%\rhead[]{}
%\lfoot[]{Luis Ángel Cruz Díaz}
%\cfoot[]{2183038433}
%\rfoot[]{\thepage}
%\renewcommand{\footrulewidth}{0.08pt}
%\renewcommand{\headrulewidth}{0pt}
% --- Para renombrar las secciones
%\renewcommand{\thesection}{\Roman{section}}
%\renewcommand{\thesubsection}{\thesection.\Roman{subsection}}
%\renewcommand{\thesubsubsection}{\thesubsection.\alph{subsubsection}}
% --- Para el manejo de resaltado de texto
\usepackage{color}
\usepackage{soul}
% --- Para el manejo de código
\usepackage{listings}
\usepackage{xcolor}
\definecolor{verde}{rgb}{0,0.6,0}
\definecolor{gris}{rgb}{0.5,0.5,0.5}
\definecolor{mageta}{rgb}{0.58,0,0.82}
\lstset{ 
  backgroundcolor=\color{white},% Indica el color de fondo; necesita que se añada \usepackage{color} o \usepackage{xcolor}
  basicstyle=\footnotesize,     % Fija el tamaño del tipo de letra utilizado para el código
  breakatwhitespace=false,      % Activarlo para que los saltos automáticos solo se apliquen en los espacios en blanco
  breaklines=true,              % Activa el salto de línea automático
  captionpos=b,                 % Establece la posición de la leyenda del cuadro de código
  commentstyle=\color{verde},   % Estilo de los comentarios
  deletekeywords={...},         % Si se quiere eliminar palabras clave del lenguaje
  extendedchars=true,           % Permite utilizar caracteres extendidos no-ASCII; solo funciona para codificaciones de 8-bits; para UTF-8 no funciona. En xelatex necesita estar a true para que funcione.
  %frame=single,	            % Añade un marco al código
  keepspaces=true,              % Mantiene los espacios en el texto. Es útil para mantener la indentación del código(puede necesitar columns=flexible).
  keywordstyle=\color{black},    % estilo de las palabras clave
  otherkeywords={*,...},        % Si se quieren añadir otras palabras clave al lenguaje
  numbers=none,                 % Posición de los números de línea (none, left, right).
  numbersep=8pt,                % Distancia de los números de línea al código
  numberstyle=\tiny\color{gris},% Estilo para los números de línea
  rulecolor=\color{black},      % Si no se activa, el color del marco puede cambiar en los saltos de línea entre textos que sea de otro color, por ejemplo, los comentarios, que están en verde en este ejemplo
  showspaces=false,             % Si se activa, muestra los espacios con guiones bajos; sustituye a 'showstringspaces'
  showstringspaces=false,       % subraya solamente los espacios que estén en una cadena de esto
  showtabs=false,               % muestra las tabulaciones que existan en cadenas de texto con guión bajo
  stepnumber=1,                 % Muestra solamente los números de línea que corresponden a cada salto. En este caso: 1,3,5,...
  stringstyle=\color{mageta},   % Estilo de las cadenas de texto
  tabsize=4,	                % Establece el salto de las tabulaciones a 2 espacios
  title=\lstname                % muestra el nombre de los ficheros incluidos al utilizar \lstinputlisting; también se puede utilizar en el parámetro caption
}
% --- Para el manejo de acentos
\lstset{literate=
  {á}{{\'a}}1 {é}{{\'e}}1 {í}{{\'i}}1 {ó}{{\'o}}1 {ú}{{\'u}}1
  {Á}{{\'A}}1 {É}{{\'E}}1 {Í}{{\'I}}1 {Ó}{{\'O}}1 {Ú}{{\'U}}1
  {à}{{\`a}}1 {è}{{\`e}}1 {ì}{{\`i}}1 {ò}{{\`o}}1 {ù}{{\`u}}1
  {À}{{\`A}}1 {È}{{\'E}}1 {Ì}{{\`I}}1 {Ò}{{\`O}}1 {Ù}{{\`U}}1
  {ä}{{\"a}}1 {ë}{{\"e}}1 {ï}{{\"i}}1 {ö}{{\"o}}1 {ü}{{\"u}}1
  {Ä}{{\"A}}1 {Ë}{{\"E}}1 {Ï}{{\"I}}1 {Ö}{{\"O}}1 {Ü}{{\"U}}1
  {â}{{\^a}}1 {ê}{{\^e}}1 {î}{{\^i}}1 {ô}{{\^o}}1 {û}{{\^u}}1
  {Â}{{\^A}}1 {Ê}{{\^E}}1 {Î}{{\^I}}1 {Ô}{{\^O}}1 {Û}{{\^U}}1
  {ã}{{\~a}}1 {ẽ}{{\~e}}1 {ĩ}{{\~i}}1 {õ}{{\~o}}1 {ũ}{{\~u}}1
  {Ã}{{\~A}}1 {Ẽ}{{\~E}}1 {Ĩ}{{\~I}}1 {Õ}{{\~O}}1 {Ũ}{{\~U}}1
  {’}{{\''}}1 {Œ}{{\OE}}1 {æ}{{\ae}}1 {Æ}{{\AE}}1 {ß}{{\ss}}1
  {ű}{{\H{u}}}1 {Ű}{{\H{U}}}1 {ő}{{\H{o}}}1 {Ő}{{\H{O}}}1
  {ç}{{\c c}}1 {Ç}{{\c C}}1 {ø}{{\o}}1 {å}{{\r a}}1 {Å}{{\r A}}1
  {€}{{\euro}}1 {£}{{\pounds}}1 {«}{{\guillemotleft}}1
  {»}{{\guillemotright}}1 {ñ}{{\~n}}1 {Ñ}{{\~N}}1 {¿}{{?`}}1 {¡}{{!`}}1 
}
% --- Para el cambio de nombre de pie de página de código	
\renewcommand{\lstlistingname}{Código}
\renewcommand{\lstlistlistingname}{Lista de \lstlistingname s}
% --- Para el manejo de referencias
\usepackage{csquotes}
\usepackage[backend=biber,citestyle=authoryear,style=apa]{biblatex}
\bibliography{referencias}
% Paquete para el manejo de apéndices
\usepackage{appendix} % Para agregar apéndices
\renewcommand{\appendixpagename}{Apéndice}
\renewcommand{\appendixtocname}{Apéndice}

\renewcommand{\theenumi}{\textbf{\alph{enumi}}}
\renewcommand{\theenumii}{\textbf{\roman{enumii}}}